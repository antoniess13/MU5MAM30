\documentclass[11pt]{article} 

\usepackage{pdfsync} 
\usepackage{url}
\usepackage{amsmath, amssymb}
\usepackage{hyperref}

\newcommand{\R}{\mathbb{R}}

\topmargin=0cm
\oddsidemargin0mm
\textheight23.5cm
\textwidth16cm
\headsep0mm
\headheight0mm
\parskip 2pt

\title{\bf 
	Sorbonne Université\\
	Implementation of finite element methods (MU5MAM30)\\[1cm]
	Some project proposals
}

\author{}
\date{}

\begin{document}
\maketitle

\section*{Instructions}

Projects can be tackled in groups of at most three students (but different groups
can work independently on the same subject). Reports should be brief and to the
point, and ideally typeset in Latex or Markdown. They must be accompanied with source
code and build instructions. They must be uploaded on the git repo in the
Project directory or sent to me by email, in the latter case in a single archive 
(zip or equivalent) with basename format 
\texttt{name\_surname-MU5MAM30-project}. {\bf Deadline for upload is
February 21st 2025}. Appointments (30min slot, on an individual basis even if working in 
group) must be taken for a project presentation between January 6th and 
February 28th 2025, by e-mail at {\tt didier.smets@sorbonne-universite.fr}.
It is accepted but not needed to prepare slides for the presentation, I will 
have read your code and report in advance,  you will explain me the difficulties 
and achievements, and I will ask you some questions related to your submission 
to evaluate your scientific autonomy.    


\medskip
In case of guidance needed, for the choice or the realization of the project, 
contact me by e-mail or show-up in my office in 16-26 325. Proposing an
alternate project is also accepted, but you must send me the proposal for
agreement first.

\pagebreak

%%%%%%%%%%%%%%%%%%%%%%%%%%%%%%%%%%%%%%%%%%%%%%%%%%%%%%%%%%%%%%%%%%%%%%%%%%%%%%%%%%%%%%%%%%%%%%%%%%%%%%%
\section*{Project 1 : Adding a boundary constraint}
The goal of this project is to modify the Navier-Stokes solver
which we have devised in class in order to tackle surface meshes 
with a boundary, with the condition that the velocity $u$ of the fluid
is tangent to the boundary.

In the vorticity/stream function framework that we have adopted,
the velocity $u$ of the fluid and the stream function $\Psi$ were related
by the equation  
$$
u = \nabla^\perp \Psi.
$$
A natural set-up in the presence of boundaries is to assume a 
non-penetrating condition, that is
$$
u \cdot n = 0 \text{ on } \partial \Omega,
$$
where here $n$ denotes the normal to the boundary. This can be rewritten as
$$
\nabla^\perp\Psi \cdot n = 0, \qquad \text{or equivalently} \qquad   \nabla \Psi
\cdot \tau = 0,
$$
where here $\tau$ is a tangent vector to the boundary. The latter implies that 
$\Psi$ is constant on (each connected component of) $\partial \Omega.$ In the project
you may assume that $\partial \Omega$ is connected (as is the case e.g. for
$\Omega$ a half-sphere surface mesh, hence only slightly modifying our class 
framework, or even more simply a 2D square grid). In that case, since $\Psi$ is 
only also globally defined up to a fixed constant (adding a constant to $\Psi$
does not modify the physical quantity $u$), you can assume that 
$\Psi \equiv 0$ on $\partial \Omega.$ Therefore, for the Poisson solver part of
the resolution, you will have
$$
\Delta \Psi = \omega \text{ on } \Omega,\qquad \Psi = 0 \text{ on } \partial
\Omega. 
$$

{\it Bonus/Suggestion if you wish to dig further :} Try discussing the 
boundary conditions for $\omega$ that we are implicitly assuming here 
by not modifying the weak formulation for $\omega$ with respect to the
case without boundary. Other more physically relevant conditions might 
impose the value of the tangential part of the velocity (e.g. a 
no-slip condition corresponding to $u \cdot \tau = 0$, which in addition to
the impermeability condition yields $u = 0$), or a so-called Navier type 
condition. Both of these are mathematically more challenging (non local) in 
the context of the vorticity-stream formulation.

{\it Coding suggestion:} You might clone (some of) my files in the base {\tt include}
and {\tt src} directories of the git repository into your own Student directory 
and then modify them there (e.g. at least {\tt src/fem/navier\_stokes.cpp}). 
That will allow you to test your modification code with graphical
feed-back\footnote{That will require cloning the {\tt extern} and {\tt shaders}
directories too.} at minimal coding cost.

\pagebreak
%%%%%%%%%%%%%%%%%%%%%%%%%%%%%%%%%%%%%%%%%%%%%%%%%%%%%%%%%%%%%%%%%%%%%%%%%%%%%%%%%%%%%%%%%%
\section*{Project 2 : Elimination tree and direct sparse Cholesky solver}

For this project you will write your own Cholesky factorization code  
$$
A = LL^T
$$
for SPD sparse matrices $A$ in the CSR format. 

One key step of this factorization in the framework of sparse matrices is to
determine in advance which 
matrix entries in $L$ might end-up being non zero (the fill-in), because these may be
much more than in $A$ and it is necessary or at least useful to ``prepare'' $L$ (reserve memory
etc) before filling it. This phase is usually called the {\it symbolic} phase, because 
the potential non zero entries are located and counted in each row, but are not
yet computed.

Since the theoretical background behind the process requires a bit of thinking, 
you are suggested to use as a reference the book {\it Algorithms for Sparse 
Linear Systems} by J. Scott \& M. Tuma (in open access, see e.g. 
\href{https://link.springer.com/book/10.1007/978-3-031-25820-6}{here} for
download). Mostly relevant to us here are sections 4.2, 4.3, (5.1) and 5.2. 
In particular you will review the notion of {\it elimination tree}, and you 
will directly translate algorithms 4.2 and 4.3 (for the symbolic phase) and 
then 5.7 (for the actual factorization) into code. For the factorization step,
it is important to avoid computing elements of $L$ which were not marked as
potential non zeros in the symbolic phase. This last phase remains the most
computationally expensive though.


{\it Bonus/Suggestion if you wish to dig further :} 
Already after the symbolic phase, you may test using your code (e.g. computing the amount of
fill-in) the importance of the ordering of the unknowns 
(recall that in the case of FEM P1 mass or stiffness matrices, unknowns simply
correspond values at mesh vertices, so it is just vertex reordering). 
For the specific and relatively simple case of the cube mesh, you can try to reorder vertices according to a nested
dissection strategy\footnote{as discussed in the before last lecture, and in section 8.4
of the book suggested above} ``by hand'' (no need to revert to complicated graph partitioning
algorithms) : take all the vertices on the edges of the cube
as first separator set, and then divide the hence formed 6 remaining 
square faces recursively and alternatively into one or another direction (for a 2D square
grid you would just cut it into two alternatively in the horizontal and vertical
directions; in 3D you simply cut it in the direction in which it is presently the
longest). 

\pagebreak
%%%%%%%%%%%%%%%%%%%%%%%%%%%%%%%%%%%%%%%%%%%%%%%%%%%%%%%%%%%%%%%%%%%%%%%%%%%%%%%%%%%%%%%%%%%%%%%%%%%%%%%%%%%%%
\section*{Project 3 : A (convex) nonlinear elliptic problem}

Let $\Omega$ be a bounded domain in $\R^2$ and let $f : \Omega \to \R.$ 
The area of the graph of $f$ is given by 
$$
A(f) := \int_\Omega \sqrt{1+ |\nabla f|^2}. 
$$
The graph of $f$ is said to be a minimal surface / minimal graph if
$$
A(f) \leq A(g) \text{ for any } g:\Omega \to \R \text{ s.t. } f=g \text{ on }
\partial \Omega,
$$
in other words if its area is minimal among all graphs assuming the same boundary curve.
A minimal graph satisfies the minimal surface equation 
$$
{\rm div}\left( \frac{\nabla f}{\sqrt{1 + |\nabla f|^2}}\right) = 0 \text{ on } \Omega,
$$
which is nothing but the Euler-Lagrange equation associated to $A(\cdot).$

Given a planar 2D triangular mesh $\Omega_h$, and piecewise affine 
function $f$ over that mesh (hence representable using a P1 Lagrange finite
element basis), present and code an algorithm that will compute/converge to the
(unique) minimal graph $f_*$ in that same finite element space and assuming
the same boundary values as $f$.\\
{\it Hint :} Note that the set of admissible candidates is an affine subspace of the P1
Lagrange space over $\Omega_h$, and the functional $A(\cdot)$ is convex on that
set. Any reasonable gradient descent method should therefore converge.
 
{\it Bonus/Suggestion if you wish to dig further :}
In nature minimal surfaces (e.g. soap films) need not be graphs. The mathematical 
set-up required to study 
the so-called Plateau's problem (finding a minimal surface given a boundary
curve) is therefore somewhat more complex, but the resulting Euler-Lagrange
equation is simpler since it amounts to the homogeneous linear Poisson equation
$\Delta f = 0.$ It comes with additional nonlinear constrains though, and
besides $f$ is no longer a scalar function (above the graph was $z = f(x,y)$, 
in the parametric formulation $f(x,y,z) = (f_1(x,y), f_2(x,y), f_3(x,y))$ is
a vectoru function. Solutions (they need no longer be unique) may still be searched 
for using a P1 finite element methods, you might for example try to reproduce the 
strategy explained in \href{https://eudml.org/doc/221286}{that} paper, in section 5.

\pagebreak
%%%%%%%%%%%%%%%%%%%%%%%%%%%%%%%%%%%%%%%%%%%%%%%%%%%%%%%%%%%%%%%%%%%%%%%%%%%%%%%%%%
\section*{Project 4 : Using existing software}

Understanding the math and the difficulties/challenges associated to the
implementation of finite element methods is important in order to
build on solid ground. Since the subject is broad, it is also helpful if not 
important to be able to reuse (and potentially improve or make evolve) existing
code bases. For this last subject proposal, the goal is to reproduce the exact 
same 2D Navier-Stokes solver on a sphere mesh that we coded from scratch, but 
only using third party libraries/software for each step (meshing, build-up of
matrices and solution of linear systems). You might choose one among the
following well established open source ones :
\begin{itemize}
\item {\bf FreeFem} \url{https://freefem.org}\\
More a software than a library, it comes with its own scripting language 
leading to short single file scripts. In particular, you won't write any 
C/C++ code if choosing it.
\item {\bf FEniCS}  \url{https://fenicsproject.org}\\
Or it's DOLPHINx Python wrapper, if you also wish to avoid C/C++.
\item {\bf Deal.II} \url{https://www.dealii.org}\\
\end{itemize}
Demos and tutorials exist for all three, they might be easier to begin with
w.r.t. reading their documentation. The first one, FreeFem, is developed 
here at SU. 






\end{document}
